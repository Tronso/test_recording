% arara: pdflatex: {action: nonstopmode}

\documentclass{scrreprt}

\usepackage{listings}
\usepackage{underscore}
\usepackage[bookmarks=true]{hyperref}
\usepackage{tabu}
\usepackage{multirow}

\hypersetup{
    bookmarks=false,    % show bookmarks bar?
    pdftitle={Requirement Specification},    % title
    pdfauthor={Pascal Grosch},                     % author
    pdfsubject={Requirements: Recording of Scenarios for Autonomous Vehicles},                        % subject of the document
    pdfkeywords={testing, autonomous vehicle, bachelor thesis, requirements, recording}, % list of keywords
    colorlinks=true,       % false: boxed links; true: colored links
    linkcolor=black,       % color of internal links
    citecolor=black,       % color of links to bibliography
    filecolor=black,        % color of file links
    urlcolor=purple,        % color of external links
    linktoc=page            % only page is linked
}%

\title{%
\flushright
\rule{16cm}{5pt}\vskip1cm
\Huge{SOFTWARE REQUIREMENTS\\ SPECIFICATION}\\
\vspace{2cm}
for\\
\vspace{2cm}
Recording System for Scenarios for Autonomous Vehicles\\
\vfill
\rule{16cm}{5pt}
}
\date{}

\usepackage{hyperref}

\tabulinesep=1.5mm

\newcommand{\goal}[4]{
	\label{#1}
	\begin{tabu}{|X[-1,m]|X[l,m]|}
		\hline
		\textbf{ID} & #1\\ \hline
		\textbf{Name} & #2\\ \hline
		\textbf{Description} & #3\\ \hline
		\textbf{Success Criterion} & #4\\ \hline
	\end{tabu}~\\
	}
	
\newcommand{\usecase}[9]{
	\label{#1}
	\begin{tabu}{|X[-1,m]|X[l,m]|}
		\hline
		\textbf{ID} & #1\\ \hline
		\textbf{Name} & #2\\ \hline
		\textbf{Precondition} & #3\\ \hline
		\textbf{Description} & #4\\ \hline
		\textbf{Steps} & #5\\ \hline
		\textbf{Postcondition} & #6\\ \hline
		\textbf{Exceptions} & #7\\ \hline
		\textbf{Quality Requirements} & #8\\ \hline
		\textbf{Links} & #9\\ \hline
	\end{tabu}
	}

\newcommand{\systemfunction}[9]{
	\label{#1}
	\begin{tabu}{|X[-1,m]|X[l,m]|}
		\hline
		\textbf{ID} & #1\\ \hline
		\textbf{Name} & #2\\ \hline
		\textbf{Input} & #3\\ \hline
		\textbf{Precondition} & #4\\ \hline
		\textbf{Description} & #5\\ \hline
		\textbf{Postcondition} & #6\\ \hline
		\textbf{Exceptions} & #7\\ \hline
		\textbf{Quality Requirements} & #8\\ \hline
		\textbf{Links} & #9\\ \hline
	\end{tabu}~\\ 
	}
	
\newcommand{\nfr}[3]{
	\label{#1}
	\begin{tabu}{|X[-1,m]|X[l,m]|}
		\hline
		\textbf{ID} & #1\\ \hline
		\textbf{Name} & #2\\ \hline
		\textbf{Description} & #3\\ \hline
	\end{tabu}~\\ 
	}
	
\newcommand{\ir}[4]{
	\label{#1}
	\begin{tabu}{|X[-1,m]|X[l,m]|}
		\hline
		\textbf{ID} & #1\\ \hline
		\textbf{Name} & #2\\ \hline
		\textbf{Description} & #3\\ \hline
		\textbf{Priority} & #4\\ \hline
	\end{tabu}~\\ 
	}
	
\newcommand{\myref}[1]{
	\hyperref[#1]{#1}
}

\begin{document}
\maketitle
\tableofcontents

\chapter{Goals}
	
	\goal{G10}
		{Record Traffic Situations}
		{The system creates an entry in the database that contains information about a sequence of traffic situations in which the VUT is unstable.}
		{The data is entered in the database and can be used to recreate the situations.}
	\\
	\goal{G20}
		{Display Recorded Data}
		{The system retrieves the stored data and displays it in a graphical format.}
		{The contents of the data storage are displayed correctly and readable by humans}

\chapter{Use Cases}

	\usecase{UC10}
		{Recording of Traffic Situations}
		{Exactly one VUT exists and is in a stable state.}
		{The simulation is started and the data is delivered to the recording framework where it is processed and stored.
		After the simulation run has ended the database contains the information about all unstable situations and transitions between them.}
		{\begin{enumerate}
			\item Execute the simulation
			\item Receive data for each step
			\item Translate the Raw data to predicates
			\item Detect situation changes
			\item Store data
		\end{enumerate}}
		{Every change of situations is stored in the database.}
		{-}
		{The recording should not slow down the simulation below a speed of 1 simulated second per real second.}
		{\myref{G10}}
	~\\ \\
	\usecase{UC20}
		{Display Recorded Situations}
		{Valid data is stored in the database.}
		{The data is retrieved from the databased and exported as an image representing the situation graph.}
		{\begin{enumerate}
			\item Retrieve Data
			\item Display situation graph
			\item Store Data
		\end{enumerate}}
		{Every situation and transition stored in the database is displayed in a file.}
		{There is no data in the database}
		{-}
		{\myref{G20}}

\chapter{Requirements}

\section{System Functions}
	
	\systemfunction{SF10}
		{Execute Simulation}
		{-}
		{A VUT exists in the simulation.}
		{The simulation is run step by step}
		{-}
		{-}
		{-}
		{\myref{UC10}}
	\\
	\systemfunction{SF20}
		{Receive Data}
		{raw data of a simulated vehicle}
		{data for all observed variables is present}
		{The raw data of a simulated vehicle is received by the data processor}
		{The data is temporarily stored.}
		{-}
		{-}
		{\myref{UC10}}
	\\
	\systemfunction{SF30}
		{Process Data}
		{raw data of a simulated vehicle}
		{data for all observed variables is present}
		{The raw data of a simulated vehicle is converted into predicate logic. Hysteresis are used to prevent jumping between two states.}
		{The data is formatted into predicate logic}
		{-}
		{-}
		{\myref{UC10}}
	\\
	\systemfunction{SF40}
		{Detect Situation Changes}
		{data formatted into predicate logic, current situation}
		{data for all observed variables is present}
		{The observed data is compared to the data in the current situation. If a difference is detected a transition between the situations is created.}
		{If the situations differ a transition is temporarily stored in the processing unit and the new situation is stored as the current situation.}
		{If no current situation is available no comparison is done.
		If both situations are stable no comparison is done.}
		{-}
		{\myref{UC10}}
	\\
	\systemfunction{SF50}
		{Store Data}
		{data formatted into predicate logic}
		{difference to current situation was found or no current situation  is present}
		{The situations and transition are translated to the database format and stored in the database.}
		{The data is stored in the database}
		{-}
		{-}
		{\myref{UC10}, \myref{UC20}}
	\\
	\systemfunction{SF60}
		{Display Data}
		{set of situations and transitions}
		{for each transition the preceding and succeeding situation is given}
		{The data is transformed into a visual directed graph with situations as nodes and transitions as edges.}
		{A human readable graph is created.}
		{The set of situations and transitions is empty.}
		{-}
		{\myref{UC20}}
	\\
	\systemfunction{SF70}
		{Retrieve Data}
		{-}
		{data is present}
		{The data corresponding to the ID is retrieved from the database and converted to the system format.}
		{A situation tree is restored in the work format.}
		{-}
		{-}
		{\myref{UC20}}

\section{Nonfunctional Requirements}
	
	\nfr{NFR10}
		{Simulation Speed}
		{The simulated speed should never fall below 1 simulated second per real second.}

\section{Inverse Requirements}
	
	There are no inverse requirements for the system.

\section{Design Decisions}

	\begin{tabu}{|X[-1,m]|X[l,m]|}
		\hline
		\rowfont[l]{\bfseries} ID & Description\\ \hline
		DD10 & SQLite Database should be integrated. The solution should foresee a change to another database, i.e. Neo4j instead of SQLite.\\ \hline
		DD20 & Since a framework to run simulations and retrieve data is already available for PTV Vissim, PTV Vissim is used as the simulator.\\ \hline
		DD30 & Since the COM-Interface and the existing framework is written in Visual C++ the system will also be written in Visual C++, using the the C++14 standard.\\ \hline
		DD40 & For the visualization of the graph Graphviz is used since it has a C/C++ library that already handles most of the structure, layout and memory management. Additionally the form of the nodes and edges can easily be adapted to different use cases.\\ \hline
	\end{tabu}

\chapter{Traceability}
	
	\begin{tabu}{|X[-1,m]|X[-1,m]|X[-1,m]|}
		\hline
		\rowfont[l]{\bfseries} Goal & Use Case & System Function\\ \hline
		\multirow{5}{*}{\myref{G10}} 	& \multirow{5}{*}{\myref{UC10}}	& \myref{SF10}\\
								& 						& \myref{SF20}\\
								&						& \myref{SF30}\\
								&						& \myref{SF40}\\
								&						&\myref{SF50}\\ \hline
		\multirow{3}{*}{\myref{G20}}	& \multirow{3}{*}{\myref{UC20}}	& \myref{SF50}\\
								& 						& \myref{SF60}\\
								&						& \myref{SF70}\\ \hline
	\end{tabu}

% add other chapters and sections to suit
\end{document}